\documentclass[main.tex]{subfiles}
\begin{document}
\section{Energy}
\subsection{System with potential energy}
\label{potential_energy}
In the considerations below, we will not distiguish between column and row vectors.
We will consider a system of particles, which positions will be denoted by
\begin{equation}
\label{generalised-trajectory}
q = 
\begin{bmatrix}
\vec{q}_1 &
\vec{q}_2 &
\dots &
\vec{q}_n 
\end{bmatrix}
\end{equation}
where
\begin{equation}
\vec{q}_i = 
\begin{bmatrix}
q^1_i & 
q^2_i & 
q^3_i
\end{bmatrix}.
\end{equation}
We consider $q$ as depentent on $t$. By $\dot{q}$ we denote $\cfrac{dq}{dt}$.


Let
\begin{equation}
F = 
\begin{bmatrix}
\vec{F}_1 & \vec{F}_2 \dots \vec{F}_n
\end{bmatrix}
\end{equation} 
where $\vec{F}_i$ is a force acting on $i$-th partcile.

Assume that we have some scalar value $V$ which dependents on $q$ and
\begin{equation}
\label{force_in_potential}
F = -\cfrac{\partial V}{\partial q}.
\end{equation}

By this we assume that for our system of partciles forces depend only on position of particles.

Assume now that our system evolve from a state $q_0$ at $t=t_0$ to a state $q_1$ at $t=t_1$. Let's try to calculate work of the forces $F$ when system changed from $q_0$ to $q_1$ for an arbitrary evolution in time $q(t)$ with this constrain only that $q(t_0) = q_0$ and $q(t_1) = q_1$.
\begin{equation}
W = \int_{t_0}^{t_1} F\cdot \cfrac{dq}{dt} dt=-\int_{t_0}^{t_1} \cfrac{\partial V}{\partial q}\cdot \cfrac{dq}{dt} dt = - \int_{t_0}^{t_1} \cfrac{dV}{dt} = V(q_0) - V(q_1).
\end{equation}
\begin{equation}
\boxed{
W = V(q_0) - V(q_1)
}
\end{equation}

That's why we call $V$ a potential energy in a state $q$. The energy that the forces of system needs to use is independent on the path of evolution. You can know this just but substracting respectively potential energy of an start poit and end point.

\subsection{Lagrangian picture}
\label{lagrange-picture}
Assume that the system from Subsection \ref{potential_energy} is described by Lagranian 
\begin{equation}
\label{lagrange_kinetic_assumption}
L = T - V.
\end{equation}
Where $T$ is a kinetic energy of the system
\begin{equation}
T = \sum_{i=1}^n \cfrac{1}{2}m_i|\dot{\vec{q}}_i|^2.
\end{equation}
And we know about $V$ only that it is dependent only on $q$.
Now, let's check what we can get from Euler-Lagrange equation
\begin{equation}
\label{lagrange_equation2}
\cfrac{\partial L}{\partial q} - \cfrac{d}{dt}\cfrac{\partial L}{\partial \dot{q}} = 0.
\end{equation}
Firstly, note that it translates into
\begin{equation}
-\cfrac{\partial V}{\partial q} - \cfrac{d}{dt}\cfrac{\partial T}{\partial \dot{q}} = 0,
\end{equation}
hence 
\begin{equation}
\cfrac{d}{dt}\cfrac{\partial T}{\partial \dot{q}} = -\cfrac{\partial V}{\partial q}.
\end{equation}
Note that
\begin{equation}
\cfrac{\partial T}{\partial \dot{\vec{q_i}}} = m_i\dot{\vec{q_i}},
\end{equation}
thus
\begin{equation}
\cfrac{d}{dt}\cfrac{\partial T}{\partial \dot{\vec{q_i}}} = m_i\ddot{\vec{q_i}} = \vec{F}_i.
\end{equation} 
Where  $\vec{F}_i$ is just a newtonian dynamical force, which can be measured just by mass times acceleration.

From that we get
\begin{equation}
F = -\cfrac{\partial V}{\partial q},
\end{equation}
which is the equation (\ref{force_in_potential}) and thus the rest of the Subsection \ref{potential_energy} applies, so we can call $V$ potential energy.

Because $L$ doesn't depend on $t$, we get
\begin{equation}
L - \cfrac{\partial L}{\partial \dot{q}} \dot{q} = \text{const}.
\end{equation}
The above follows imediatelly from Noether Theorem (Theorem \ref{noether}) but also can be derived directly from calculating $\frac{d}{dt}(L - \frac{\partial L}{\partial \dot{q}} \dot{q})$ and applying (\ref{lagrange_equation2}) to $\frac{dL}{dt}$.
\begin{equation}
\label{energy-coserv-simple}
\cfrac{\partial L}{\partial \dot{q}} \dot{q} = \cfrac{\partial T}{\partial \dot{q}} \dot{q} = \sum_{i=1}^n \cfrac{\partial T}{\partial \dot{\vec{q_i}}} \cdot \dot{\vec{q_i}} = \sum_{i=1}^n m_i|\dot{\vec{q}}_i|^2 = 2T.
\end{equation}

Thus $L - 2T = \text{const}$, from which by (\ref{lagrange_kinetic_assumption}) follows imediatelly
\begin{equation}
T + V = \text{const}.
\end{equation}
Which is energy conservation principle.
\subsection{Hamiltonian picture}
Assume that the system from from Subsection \ref{potential_energy} is described by Hamiltonian
\begin{equation}
\label{hamilton-kinematic-assumption}
H = T + V,
\end{equation}
where $V$ is defined like in Subsection \ref{lagrange-picture} and
\begin{equation}
T = \sum_{i=1}^n \cfrac{1}{2m_i}|\vec{p}_i|^2.
\end{equation}
System satisfies Hamilton's equations:
\begin{equation}
\cfrac{\partial H}{\partial q} = -\dot{p},
\end{equation}
\begin{equation}
\label{hamilton-second-1}
\cfrac{\partial H}{\partial p} = \dot{q}.
\end{equation}
Let's try to calculate $\ddot{q}$.
\begin{multline}
\\
\ddot{\vec{q_i}} = \cfrac{d}{dt}\cfrac{\partial H}{\partial \vec{p_i}} = \cfrac{\partial^2 H}{\partial \vec{p_i}\partial t}+ \cfrac{\partial^2 H}{\partial \vec{p_i}\partial q}\cfrac{dq}{dt}+\cfrac{\partial^2 H}{\partial \vec{p_i}\partial p}\cfrac{dp}{dt}\\
= \cfrac{\partial^2 H}{\partial \vec{p_i}\partial p}\dot{p} = \cfrac{\partial^2 H}{\partial \vec{p_i}\partial \vec{p_i}}\dot{\vec{p_i}} = \cfrac{1}{m_i} \dot{\vec{p_i}}.
\end{multline}
Thus
\begin{equation}
m_i \ddot{\vec{q_i}} = \dot{\vec{p_i}}.
\end{equation}
Therefore, from definition
$\vec{F_i} = \dot{\vec{p_i}}$ and $F = \dot{p}$.
Hence, from (\ref{hamilton-second-1})
\begin{equation}
F = -\cfrac{\partial V}{\partial q},
\end{equation}
which is the equation (\ref{force_in_potential}) and thus the rest of the Subsection \ref{potential_energy} applies, so we can call $V$ potential energy.
Note that
\begin{equation}
\cfrac{d}{dt} H = \cfrac{\partial H}{\partial t} + \cfrac{\partial H}{\partial q}\dot{q} + \cfrac{\partial H}{\partial p}\dot{p} = \cfrac{\partial H}{\partial q}\cfrac{\partial H}{\partial p} - \cfrac{\partial H}{\partial p}\cfrac{\partial H}{\partial q} = 0.
\end{equation}
Thus $H = \text{const}$ and it imediatelly follows from (\ref{hamilton-kinematic-assumption}) that
\begin{equation}
T + V = \text{const}.
\end{equation}
Which is energy conservation principle.

\section{Euler-Lagrange equation}

We will give now more abstract treatment of classical mechanics. We will still use generalised trajectory $q$ as defined in (\ref{generalised-trajectory}). 
We will assume we have a real value $L$ which depends on values of generalised vectors $q$, $\dot{q}$ and $t$. It is our intention to show how $L$ encodes the whole mechanics of the system.

Let's define an action functional
\begin{equation}
\label{action-functional-def}
S(t_0, q_0, t_1, q_1, q) \stackrel{def}{=} \int_{t_0}^{t_1} L(q(t), \dot{q}(t), t) dt,
\end{equation}

where $q_0$ is a generilised starting position of system of particles, $q_1$ is a generalised ending poistion of system of particles, $t_0$ is start time, $t_1$ is end time and by $q$ we denote an arbitrary trajectory for which $q_0 = q(t_0)$ and $q_1 = q(t_1)$. Note that by trajectory we understand positions in function of time, then in this sense trajectory encodes both position in given time and velocity as first derivative of $q$.

It should be noted that $q$ in equations is slighlty ambigous. Once it denotes trajectory, other time position.
\\
\\
\indent
We define our mechanics by assumtion that only those trajectories are ``allowed" which are stationary trajectories of $S$ for established $t_0, q_0, t_1, q_1$. Trajectory $q$ is stationary in the above sense if for any infinitesimal variation $\delta q$ such that $\delta q(t_0) = \delta q(t_1) = 0$, we have $\delta S \stackrel{def}{=} S(q + \delta q) - S(q) = 0$ (i.e. difference in $S$ is 0 up to order of magnitute of $\delta q$).
\\
\\
\indent
Let us calculate $\delta S$ in case of general infinitesimal variation $\delta q$ not necesarily assuming that $\delta q(t_0) = \delta q(t_1) = 0$ (we will deal with this assumption later).

\begin{multline}
\\
\delta S =  \int_{t_0}^{t_1} \cfrac{\partial L}{\partial q} \delta q + \cfrac{\partial L}{\partial \dot{q}} \delta \dot{q}dt =  \int_{t_0}^{t_1} \big(\cfrac{\partial L}{\partial q} - \cfrac{d}{dt} \cfrac{\partial L}{\partial \dot{q}}\big)\delta q + \cfrac{d}{dt}\big(\cfrac{\partial L}{\partial \dot{q}} \delta q\big)dt = \\
= \int_{t_0}^{t_1} \big(\cfrac{\partial L}{\partial q} - \cfrac{d}{dt} \cfrac{\partial L}{\partial \dot{q}}\big)\delta qdt + \cfrac{\partial L}{\partial \dot{q}} \delta q \bigg|^{t_1}_{t_0}.
\end{multline}

Thus, we got
\begin{equation}
\label{variation-major}
\boxed
{
\delta S = \int_{t_0}^{t_1} \big(\cfrac{\partial L}{\partial q} - \cfrac{d}{dt} \cfrac{\partial L}{\partial \dot{q}}\big)\delta qdt + \cfrac{\partial L}{\partial \dot{q}} \delta q \bigg|^{t_1}_{t_0}
}
\end{equation}

Now, for an arbitrary infinitesimal variation which satisfies $\delta q(t_0) = \delta q(t_1) = 0$ the equation (\ref{variation-major}) becomes:

\begin{equation}
\delta S = \int_{t_0}^{t_1} \big(\cfrac{\partial L}{\partial q} - \cfrac{d}{dt} \cfrac{\partial L}{\partial \dot{q}}\big)\delta qdt.
\end{equation}

Hence if we want $\delta S = 0$ we can deduce from $\delta q$ being arbitrary that the stationary trajectory $q$ needs to saitsfy the following equation:
\begin{equation}
\cfrac{\partial L}{\partial q} - \cfrac{d}{dt} \cfrac{\partial L}{\partial \dot{q}} = 0.
\end{equation}

This is Euler-Lagrange equation.

\section{Conservation Principles}
In this section we will replace $\vec{q}_i$ notation with $\bs{q}_i$ notation, as the latter seems to be more convenient. In this sense our generic trajectory $q$ of $N$ particles is 
\begin{equation}
q = [\bs{q}_1 \dots \bs{q}_N],
\end{equation}
where $\bs{q}_i$ are time dependent.
Consider an infinitesimal symmetry transformation of $\R^3$
\begin{equation}
\bs{x} \mapsto \bs{x} + \delta\bs{x},
\end{equation}
where in a genric form (written in Einstein's summation convention)
\begin{equation}
\delta x^n = \delta\epsilon(G^n_m x^m + a^n)
\end{equation}
with $G$ is a generator of a certain symmetry group.
This infinitesimal symmetry shifts the whole system of particles in direction $-\delta \bs{x}$.
Under these shift the trajectories will transform in the following way
\begin{equation}
q \mapsto q + \delta q,
\end{equation} 

where 
\begin{equation}
\delta q^n_i = \delta\epsilon(G^n_m q^m_i + a^n).
\end{equation}

We will show now how the assumption that lagrangian $L$ of the system remains invariant under this transformation leads to certain conservation principles.

Assume that 

\begin{equation}
L(q, \dot{q}, t) - L(q + \delta q, \dot{q} + \delta \dot{q}, t) = 0.
\end{equation}

Thus
\begin{equation}
\label{conservation_stage1}
\cfrac{\partial L}{\partial q}\delta q + \cfrac{\partial L}{\partial \dot{q}}\delta \dot{q} = 0. 
\end{equation}
Recall Euler-Lagrange equation
\begin{equation}
\label{lagrange_equation3}
\cfrac{\partial L}{\partial q} - \cfrac{d}{dt}\cfrac{\partial L}{\partial \dot{q}} = 0.
\end{equation}
Thus we know that $\cfrac{\partial L}{\partial q} = \cfrac{d}{dt}\cfrac{\partial L}{\partial \dot{q}}$ and we substitute $\cfrac{\partial L}{\partial q}$ in (\ref{conservation_stage1}) accordingly obtaining:
\begin{equation}
\cfrac{d}{dt}\cfrac{\partial L}{\partial \dot{q}}\delta q + \cfrac{\partial L}{\partial \dot{q}}\delta \dot{q} = 0. 
\end{equation}
But this is the same as
\begin{equation}
\cfrac{d}{dt}\big( \cfrac{\partial L}{\partial \dot{q}}\delta q \big) = 0,
\end{equation}
from which follows a conservation principle
\begin{equation}
\label{conservation_principle_main}
\boxed{
 \cfrac{\partial L}{\partial \dot{q}}\delta q = \text{const}
 }
\end{equation}
\subsection{Canonical Momentum Conservation}
Assume that $L$ is invariant over translations of $\R^3$. Thus (\ref{conservation_principle_main}) holds for
\begin{equation}
\delta q^{n}_i = \delta\epsilon a^{n}
\end{equation}
where $\bs{a}$ has $a^n = 0$ for all $n\not=n_0$ and $a^{n_0} = 1$, where $n_0$ is an arbitrary coordinate $n_0 = 1, 2, 3$.
Hence
\begin{equation}
\sum_{i=1}^{N} \cfrac{\partial L}{\partial \dot{q}^{n_0}_i} = \text{const}.
\end{equation}
along each coordinate $n_0 = 1, 2, 3$. Thus
\begin{equation}
\boxed{
\sum_{i=1}^{N} \cfrac{\partial L}{\partial \dot{\bs{q}}_i} = \text{const}.
}
\end{equation}

The expression $\cfrac{\partial L}{\partial \dot{\bs{q}}_i}$ is called a \textit{conjugate} or \textit{canonical} momentum of an $i-th$ partcile of a system.

Note that for a particular example of Lagrangian from \ref{lagrange_kinetic_assumption},
we have 
\begin{equation}
\cfrac{\partial L}{\partial \dot{\bs{q}}_i} = m_i \bs{q}_i.
\end{equation}

Hence, under the assumption that potential $V$ is invariant under space translation (e.g. for gravitational multiparticle potential $V = \cfrac{1}{2}\sum_{i \not=j} G \cfrac{m_i m_j}{|\bs{q}_i - \bs{q}_j|}$), we have

\begin{equation}
\sum_{i=1}^{N}  m_i \bs{q}_i = \text{const}.
\end{equation} 

This is of course a classical mechanical momentum conservation principle.
\subsection{Energy Conservation Principle}
We will show now how the assumption that Lagrangian $L$ doesn't depend on time leads to energy conservation principle.
Assume that $\cfrac{\partial L}{\partial t} = 0$.
In general, we have
\begin{equation}
\cfrac{d}{dt} L = \cfrac{\partial L}{\partial q}\dot{q} + \cfrac{\partial L}{\partial \dot{q}}\ddot{q} + \cfrac{\partial L}{\partial t},
\end{equation}
but because of our assumption, we get
\begin{equation}
\cfrac{d}{dt} L = \cfrac{\partial L}{\partial q}\dot{q} + \cfrac{\partial L}{\partial \dot{q}}\ddot{q}.
\end{equation}

Because of Lagrange equation, we have $\cfrac{\partial L}{\partial q} = \cfrac{d}{dt}\cfrac{\partial L}{\partial \dot{q}}$ and thus we can substitute $\cfrac{\partial L}{\partial q}$ in the above equation accordingly obtaining:

\begin{equation}
\cfrac{d}{dt} L = \cfrac{d}{dt}\cfrac{\partial L}{\partial \dot{q}}\dot{q} + \cfrac{\partial L}{\partial \dot{q}}\ddot{q},
\end{equation}

which simplifies to
\begin{equation}
\cfrac{d}{dt} L = \cfrac{d}{dt}(\cfrac{\partial L}{\partial \dot{q}}\dot{q}),
\end{equation}

and thus
\begin{equation}
\cfrac{d}{dt} \big( \cfrac{\partial L}{\partial \dot{q}}\dot{q} - L \big) = 0.
\end{equation}
 
And this gives us a conservation principle
\begin{equation}
\boxed{
\cfrac{\partial L}{\partial \dot{q}}\dot{q} - L = \text{const}
}
\end{equation}
We will call $\cfrac{\partial L}{\partial \dot{q}}\dot{q} - L$ an energy of a system of paricles with Lagrangian $L$.

Note that we obtained this ealier for a particular example of Lagrangian (\ref{lagrange_kinetic_assumption})
by calculations in (\ref{energy-coserv-simple}). In that case we had
\begin{equation}
\cfrac{\partial L}{\partial \dot{q}}\dot{q} - L = T + V,
\end{equation}
where $T$ was a kinetic energy of a system and $V$ its potential energy.
Recall that the mentioned Lagranian does not depend on time which is in perfect agreement with our generic deriviation.

\subsection{Hamilton–Jacobi equation}
Let's analyse an action functional $S$ as defined in (\ref{action-functional-def}) but now we will be only interested in action along stationary trajectories $q$. That's why we will see $S$ as dependent only on $t_0, q_0, t_1, q_1$. For the purpouse of notation we will replace $t_1$ and $q_1$ by $t$ and $q$. This will itroduce ambiguaty between $q$ as trajectory and $q$ as ending point of trajectory $q=q(t)$, but this ambiguity is not problematic if we remeber about it.

Let's fix $q_0$ and $t_0$. In this context $S$ depends on end poition $q$ and end time $t$. Consider an infinitesimal viariation $\delta q$ of trajectory $q$ such that $\delta q(t_0) = 0$ where $q + \delta q$ is also stationary.

Let's quote the equation (\ref{variation-major})
\begin{equation}
\delta S = \int_{t_0}^{t} \big(\cfrac{\partial L}{\partial q} - \cfrac{d}{dt'} \cfrac{\partial L}{\partial \dot{q}}\big)\delta qdt' + \cfrac{\partial L}{\partial \dot{q}} \delta q \bigg|^{t}_{t_0}.
\end{equation}

Because trajectory $q$ is stationary, Euler-Lagrange equation holds and we have
\begin{equation}
\delta S = \cfrac{\partial L}{\partial \dot{q}} \delta q \bigg|^{t}_{t_0} = \cfrac{\partial L}{\partial \dot{q}} \delta q(t).
\end{equation} 

If we see $S$ as dependedn on $t$ and $q$, then $dq = \delta q(t)$ and thus
\begin{equation}
\label{canonical-momentum-for-action}
\boxed{
\cfrac{\partial S}{\partial q} = \cfrac{\partial L} {\partial \dot{q}}\
}
\end{equation}

Analise $S$ along stationary trajectory $q$. Then we have
\begin{equation}
\cfrac{dS}{dt} = \cfrac{\partial S}{\partial q}\dot{q} + \cfrac{\partial S}{\partial t}.
\end{equation}

On the other hand 
\begin{equation}
\cfrac{d S}{d t} = L.
\end{equation}

Thus
\begin{equation}
\cfrac{\partial S}{\partial t} = L - \cfrac{\partial S}{\partial q}\dot{q} = L - \cfrac{\partial L} {\partial \dot{q}} \dot{q} = -H,
\end{equation}
where $H$ is a Hamiltonian (if exists). 
Then, when hamiltonian $H(q, p, t)$ exists action functional $S$ satisfies the following equation, known as Hamilton-Jacobi equation:
\begin{equation}
\label{hamilton-jacobi-equation}
\boxed{
-\cfrac{\partial S}{\partial t} = H(q, \cfrac{\partial S}{\partial q}, t)
}
\end{equation}

\section{Reduced Mass}
Consider Lagrangian
\begin{equation}
\label{lagrangian-dist-dependent}
L(q, \dot{q}) = \cfrac{1}{2}m_1 (\dot{\vec{q_1}})^2 + \cfrac{1}{2}m_2 (\dot{\vec{q_2}})^2 + V(|\vec{q_1} - \vec{q_2}|).
\end{equation}
We apply convention similiar as in Subsection \ref{potential_energy}, where $q=[\vec{q_1}, \vec{q_2}]$.
Recall that we say that, $q_0$ extremises $L$, if $\int_{t_0}^{t_1} L(q_0, \dot{q_0})$ is a local extremum for all evolutions $q$ for which $q(t_0) = q_0(t_0)$ and $q(t_1) = q_0(t_1)$ for any moments of time $t_0, t_1$.

Take any constant velocity $\vec{v}$ and constant ponit $c_0$. Let
\begin{equation}
\label{inertial-coordinates-change}
\vec{x_i} := \vec{q_i} + t\vec{v} + c_0 \text{ for } i=1,2.
\end{equation}
Following our convention $x=[\vec{x_1}, \vec{x_2}]$. First, we will show that 
\begin{fact}
$x$ extremises $L(x, \dot{x})$ if and only if $q$ extremises $L(q, \dot{q})$.
\end{fact}

Obviously $V$ part of $L$ stays the same as $\vec{q_1} - \vec{q_2} = \vec{x_1} - \vec{x_2}$. Let's calculate
\begin{multline}
\cfrac{1}{2}m_1 (\dot{\vec{x_1}})^2 + \cfrac{1}{2}m_2 (\dot{\vec{x_2}})^2 = \\
\cfrac{1}{2}m_1 (\dot{\vec{q_1}})^2 + \cfrac{1}{2}m_2 (\dot{\vec{q_2}})^2 + (m_1\dot{\vec{q_1}} + m_2\dot{\vec{q_2}})\cdot \vec{v} + (m_1 + m_2)(\vec{v})^2.
\end{multline}
Assume that $q$ maximises $L$. From Noether Theorem, we know that $\cfrac{\partial L}{\partial \dot{q}} = \text{const}$, which is a momentum conservation principle, thus $m_1\dot{\vec{q_1}} + m_2\dot{\vec{q_2}}=\text{const}$. Hence,
\begin{equation}
\cfrac{1}{2}m_1 (\dot{\vec{x_1}})^2 + \cfrac{1}{2}m_2 (\dot{\vec{x_2}})^2 = \cfrac{1}{2}m_1 (\dot{\vec{q_1}})^2 + \cfrac{1}{2}m_2 (\dot{\vec{q_2}})^2 + \text{const}.
\end{equation}
And from that 
\begin{equation}
L(q,\dot{q}) = L(x,\dot{x}) + \text{const}.
\end{equation}
From the above $x$ extremises $L$. The situation is symetric if we want to show that $x$ extremises $L$ implies $q$ extremises $L$.
This means that for Lagrangian of type (\ref{lagrangian-dist-dependent}) any change of choordinates between frames of reference moving with constant velocity relative to each other does not change the Lagrangian of the system.

From the statement above follows that without loss of generality in case of the Lagrangian of type (\ref{lagrangian-dist-dependent}), we can always assume that the center of the mass is in the point $(0,0,0)$, i.e.
\begin{equation}
\label{mass-center-in-rest}
m_1\vec{q_1} + m_2\vec{q_2} = 0
\end{equation}

Now, if we take
\begin{equation}
\label{r-vector-diff}
\vec{r} := \vec{q_1} - \vec{q_2}, 
\end{equation}
from equations (\ref{mass-center-in-rest}) and (\ref{r-vector-diff}), we have
\begin{equation}
\begin{cases}
\vec{q_1} = \cfrac{m_2}{m_1 + m_2}\vec{r},\\
\vec{q_2} = - \cfrac{m_1}{m_1 + m_2}\vec{r}.
\end{cases}
\end{equation}
Now, after simple calculation
\begin{equation}
L = \cfrac{1}{2}\cfrac{m_1m_2}{m_1 + m_2}(\dot{\vec{r}})^2 + V(|\vec{r}|). 
\end{equation}
Hence $\vec{r}$ evolves in the same way as one particle of mass
\begin{equation}
\boxed{
m = \cfrac{m_1m_2}{m_1 + m_2}
}
\end{equation}
In a central force field $\vec{F} = - \cfrac{\partial V}{\partial \vec{r}}$. We call $m$ a reduced mass.
\section{Statistical Mechanics}
\subsection{Flux}
Let $\rho(t,x)$ will be a density of a certain abstract continous entity in time $t$ and point $x\in\reals^n$. Let $\vec{v}(t,x)$ will be a velocity of an infinitesimal element of the continous entity in time $t$ at point $x\in\reals^n$.
The current of the continous entity is
\begin{equation}
\vec{j} = \rho \vec{v}.
\end{equation}
Imagine an infinitesimal $n-1$-dimentional almost hyperplanar surfice elment $dS$ with it's normal unit vector $\vec{n}$. We will try to calculate an amount of the continous entity $dm$ which flew through surface $dS$ in the direction of the arrow of $\vec{n}$ during infenitesimal time $\Delta t$.
 
If we consider an infenitesimal element of the entity, the component of its movement which is in a plane of $dS$ doesn't play any role in its going through surface element $dS$. Since $\vec{v}\cdot \vec{n}$ is a component of veliocity $\vec{v}$ which is perpendicular to $dS$, this is the velocity with which the entity passes through $dS$. Thus
\begin{equation}
\Delta m = \rho dS\Delta t(\vec{v}\cdot\vec{n}).
\end{equation}
Hence 
\begin{equation}
\Delta m = \vec{j}\cdot\vec{n} dS\Delta t.
\end{equation}
If we take any nice enough connected open $\Omega\subset \reals^n$ with compact clousure.
We assume that entity is conserved in time, which simply means that it's not disapearing anywhere. Thus, the entity gain inside $\Omega$ during time $\Delta t$ must be equal to the negative total flux of the entity through $\partial \Omega$. Thus
\begin{equation}
\int_\Omega \rho(t + \Delta t, x)dx - \int_\Omega \rho(t, x)dx = 
-\int_{\partial \Omega} \Delta m = -\Delta t \int_{\partial \Omega} \vec{j}\cdot\vec{n} dS.
\end{equation}
Hence
\begin{equation}
\int_\Omega \cfrac{\partial \rho}{\partial t} dx = -\int_{\partial \Omega} \vec{j}\cdot\vec{n} dS.
\end{equation}
Note that the reasoning above has sense, because by continuity of $\rho$ the change of $\rho$ during time $\Delta t$ is infinitesimal while at the same time $$\cfrac{\rho(t + \Delta t, x) - \rho(t, x)}{\Delta t} = \cfrac{\partial \rho}{\partial t}(x,t)$$ is a significant value because of differentiability of $\rho$ over $t$. We need as well assume continuity of $\cfrac{\partial \rho}{\partial t}(x,t)$ to be able to go with diffrentiation under the intergral.

Now by Theorem \ref{gauss-theorem-multidimensional} (Gauss's Theorem), we have
\begin{equation}
\int_\Omega \cfrac{\partial \rho}{\partial t}dx = -\int_\Omega \nabla \cdot \vec{j}dx.
\end{equation}
Since $\Omega$ can be arbitrary, we have
\begin{equation}
\boxed{\cfrac{\partial \rho}{\partial t} = -\nabla \cdot \vec{j}}
\end{equation}
This is the relation between current and density of any continous entity which is conserved in time.

\subsection{Liouville's equation}
\label{liouville-equation}
Consider a phase space with hamiltonian $H$. Let's recall Hamilton equations
\begin{equation}
\cfrac{\partial H}{\partial x} = -\dot{p},
\end{equation}
\begin{equation}
\cfrac{\partial H}{\partial p} = \dot{x}.
\end{equation}
For convenience, in phase space we usually use Poisson brackets
\begin{equation}
\{f, g\} := \sum_i \cfrac{\partial f}{\partial x_i}\cfrac{\partial g}{\partial p_i} - \cfrac{\partial g}{\partial x_i}\cfrac{\partial f}{\partial p_i}. 
\end{equation}
Let $\vec{v} = (\dot{x}, \dot{p}) = (\cfrac{\partial H}{\partial p}, -\cfrac{\partial H}{\partial x})$ be a veliocity of a point in phase space. Assume that we have certain probability density of position and momentum $f(t, x, p)$.
Let $\vec{j} = f(t, x, p) \vec{v}$ be a current of probability. We assume that probability is conserved in time. Which means
\begin{equation}
\cfrac{\partial f}{\partial t} = -\nabla \cdot \vec{j}.
\end{equation}
Now,
\begin{multline}
\nabla \cdot \vec{j} = (\cfrac{\partial}{\partial x}, \cfrac{\partial}{\partial p}) \cdot (f\cfrac{\partial H}{\partial p}, -f\cfrac{\partial H}{\partial x}) = \\
\cfrac{\partial f}{\partial x} \cdot\cfrac{\partial H}{\partial p} + f\cfrac{\partial H}{\partial x} \cdot \cfrac{\partial H}{\partial p} - \cfrac{\partial f}{\partial p} \cdot\cfrac{\partial H}{\partial x} - f\cfrac{\partial H}{\partial p} \cdot \cfrac{\partial H}{\partial x} = \\
\cfrac{\partial f}{\partial x} \cdot\cfrac{\partial H}{\partial p} - \cfrac{\partial f}{\partial p} \cdot\cfrac{\partial H}{\partial x} =
\{f, H\}.
\end{multline}
Thus the time evolution of probability density is described by Liouville's equation
\begin{equation}
\boxed{
\cfrac{\partial f}{\partial t} = - \{f, H\}
}
\end{equation}


\end{document}