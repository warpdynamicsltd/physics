\documentclass[main.tex]{subfiles}
\begin{document}
\section{Basic properties}
\subsection{Preliminaries}
$\eta =
\left[
\begin{array}{cccc}
1 & 0 & 0 & 0 \\
0 & -1 & 0 & 0 \\
0 & 0 & -1 & 0 \\
0 & 0 & 0 & -1
\end{array}
\right] 
$
$\delta =
\left[
\begin{array}{cccc}
1 & 0 & 0 & 0 \\
0 & 1 & 0 & 0 \\
0 & 0 & 1 & 0 \\
0 & 0 & 0 & 1
\end{array}
\right] 
$
\subsection{Tensors}
$(\hat{T})^{i_1\dots i_n}_{i_{n+1}\dots i_m} =
\frac{\partial \hat{x}^{i_1}}{\partial x^{j_1}}
\cdots
\frac{\partial \hat{x}^{i_n}}{\partial x^{j_n}}
\frac{\partial x^{j_{n+1}}}{\partial \hat{x}^{i_{n+1}}}
\cdots
\frac{\partial x^{j_m}}{\partial \hat{x}^{i_m}}
T^{j_1\dots j_n}_{j_{n+1}\dots j_m}$

\noindent
$(\hat{A})^\mu =  \cfrac{\partial \hat{x}^\mu}{\partial x^\nu}A^\nu$

\subsection{Metric Tensor}
$$
\boxed{d\tau^2 =  g_{\mu\nu} dx^\mu dx^\nu}
$$

\begin{theorem}
$g_{\mu\nu}$ is a tensor.
\end{theorem}
\begin{proof}
Notice that $dx^\mu = \frac{\partial x^\mu}{\partial \hat{x}^\alpha} d\hat{x}^\alpha$.\\
Hence $d\tau^2 = g_{\mu\nu} dx^\mu dx^\nu = (g_{\mu\nu} \frac{\partial x^\mu}{\partial \hat{x}^\alpha}\frac{\partial x^\nu}{\partial \hat{x}^\beta}) d\hat{x}^\alpha d\hat{x}^\beta$.
\end{proof}

\begin{theorem}
For each point $\omega$ there exists a frame of reference 
$$(x^0, x^1, x^2, x^3)$$ 
such that $g_{\mu\nu} = \eta_{\mu\nu}$ at $\omega$.
\end{theorem}

$$
\boxed{g_{\mu\lambda}g^{\lambda\nu} = g^\nu_\mu = \delta^\nu_\mu}
$$

\subsection{Christoffel symbols}
$$
\boxed{\Gamma^\lambda_{\mu\nu} = \frac{1}{2} g^{\lambda\rho} \left( \frac{\partial g_{\rho\mu}}{\partial x^\nu} + \frac{\partial g_{\rho\nu}}{\partial x^\mu} - \frac{\partial g_{\mu\nu}}{\partial x^\rho} \right)}
$$

\subsection{Geodecis}
$$
\boxed{\cfrac{d^2\phi^\mu}{d\tau^2} + \Gamma^\mu_{\alpha\beta}\cfrac{d\phi^\alpha}{d\tau}\cfrac{d\phi^\beta}{d\tau} = 0}
$$

\subsection{Motionless particle}
Assume that we have a frame of reference $(t = x_0, x_1, x_2, x_3)$. Let's consider a motionless particle
$$(\phi^0(t) = t, \phi^1(t), \phi^2(t), \phi^3(t)). $$
As the particle is motionless, we have $\cfrac{d\phi^n}{dt} = 0$ for $n=1,2,3$ at the moment $t=0$. The particle is motionless in our space frame of reference $(x,y,z)$ at the moment $t=0$, but we assume that it's still following geodesic in the space-time.
$$
0 = \cfrac{d^2\phi^\mu}{d\tau^2} + \Gamma^\mu_{\alpha\beta}\cfrac{d\phi^\alpha}{d\tau}\cfrac{d\phi^\beta}{d\tau} = \cfrac{d^2\phi^\mu}{d\tau^2} + \Gamma^\mu_{\alpha\beta}\cfrac{d\phi^\alpha}{dt}\cfrac{d\phi^\beta}{dt}\Big(\cfrac{dt}{d\tau}\Big)^2 = \cfrac{d^2\phi^\mu}{d\tau^2} + \Gamma^\mu_{00}\Big(\cfrac{dt}{d\tau}\Big)^2.
$$

$$
\cfrac{d^2\phi^\mu}{d\tau^2} = -\Gamma^\mu_{00}\Big(\cfrac{dt}{d\tau}\Big)^2.
$$
The above holds in the point $(0, \phi^1(0), \phi^2(0), \phi^3(0))$.
$$
\cfrac{d^2\phi^n}{d\tau^2} = \cfrac{d}{d\tau}\Big(\cfrac{d\phi^n}{dt}\cfrac{dt}{d\tau}\Big) = 
\cfrac{d}{d\tau}\Big(\cfrac{d\phi^n}{dt}\Big)\cfrac{dt}{d\tau} + \cfrac{d\phi^n}{dt}\cfrac{d^2t}{d\tau^2} = 
\cfrac{d}{dt}\Big(\cfrac{d\phi^n}{dt}\Big)\Big(\cfrac{dt}{d\tau}\Big)^2 = \cfrac{d^2\phi^n}{dt^2}\Big(\cfrac{dt}{d\tau}\Big)^2
$$
for $n=1,2,3$. Thus,
$$
\cfrac{d^2\phi^n}{dt^2} = -\Gamma^n_{00} \text{ for } n=1,2,3.
$$
Therefore, 
$$
\cfrac{d^2\phi^n}{dt^2} = -\Gamma^n_{00} = -\frac{1}{2} g^{n\rho} \left( \frac{\partial g_{\rho 0}}{\partial x^0} + \frac{\partial g_{\rho 0}}{ \partial x^0} - \frac{\partial g_{0 0}}{\partial x^\rho} \right).
$$
Now we will assume that the curvature is constant in time.
\begin{assumption}
$\cfrac{\partial g_{\mu\nu}}{\partial x^0} = 0$. 
\end{assumption}
Now,
$$
\cfrac{d^2\phi^n}{dt^2} = \frac{1}{2} g^{nm}\frac{\partial g_{0 0}}{\partial x^m}.
$$
\subsection{The Ricci tensor}
$$
\boxed{
R_{\mu\nu} = \cfrac{\partial \Gamma^\alpha_{\mu\alpha}}{\partial x^\nu}
- \cfrac{\partial \Gamma^\alpha_{\mu\nu}}{\partial x^\alpha}
- \Gamma^\alpha_{\mu\nu}\Gamma^\beta_{\alpha\beta}
+ \Gamma^\alpha_{\mu\beta}\Gamma^\beta_{\nu\alpha}
}
$$

Einstein assumption for empty space:
$$
\boxed{R_{\mu\nu} = 0}
$$

\subsection{A slow particle in weakly curved empty spacetime: The Newtonian aproximation}
Let's consider a slow particle
$$(\phi^0(t) = t, \phi^1(t), \phi^2(t), \phi^3(t)).$$ Assume that $\phi^1(0) = \phi^2(0) = \phi^3(0) = 0$ and that for our frame of refrence $g_{\mu\nu} = \eta_{\mu\nu}$ in $(0,0,0,0)$. The particle is slow so we assume:
\begin{assumption}
$\cfrac{d\phi^n}{dt}$ is an infinitesimal of the first order for $n=1,2,3.$
\end{assumption}
We will asume that curvature is constant in time.
\begin{assumption}\label{con_time}
$\cfrac{\partial g_{\mu\nu}}{\partial x^0} = 0$. 
\end{assumption}

We will assume also that the curvature is weak.
\begin{assumption}
$\cfrac{\partial g_{\mu\nu}}{\partial x^n}$ is an infinitesimal of the first order for $n=1,2,3$.
\end{assumption}

\begin{proposition}\label{weak_curv}
$g_{\mu\nu} = \eta_{\mu\nu} + \epsilon_{\mu\nu}$, where $\epsilon_{\mu\nu}$ is an infinitesimal of the second order in reasonable range that we care about. 
\end{proposition}
\begin{proof}
$$
g_{\mu\nu}(d\phi) = \eta_{\mu\nu} + \cfrac{\partial g_{\mu\nu}}{\partial x^n}d\phi^n
$$ 
\end{proof}

The particle is following the geodecis in the spacetime, so with neglecting second-order infinitesimals we have:
$$
0 = \cfrac{d^2\phi^\mu}{d\tau^2} + \Gamma^\mu_{\alpha\beta}\cfrac{d\phi^\alpha}{d\tau}\cfrac{d\phi^\beta}{d\tau} = \cfrac{d^2\phi^\mu}{d\tau^2} + \Gamma^\mu_{\alpha\beta}\cfrac{d\phi^\alpha}{dt}\cfrac{d\phi^\beta}{dt}\Big(\cfrac{dt}{d\tau}\Big)^2 = \cfrac{d^2\phi^\mu}{d\tau^2} + \Gamma^\mu_{00}\Big(\cfrac{dt}{d\tau}\Big)^2.
$$


Thus

\begin{equation}\label{acce}
\cfrac{d^2\phi^\mu}{d\tau^2} = -\Gamma^\mu_{00}\Big(\cfrac{dt}{d\tau}\Big)^2.
\end{equation}

Putting $\mu = 0$ we may conclude that $\cfrac{d^2t}{d\tau^2}$ is a first order infinitesimal.

$$
\cfrac{d^2\phi^n}{d\tau^2} = \cfrac{d}{d\tau}\Big(\cfrac{d\phi^n}{dt}\cfrac{dt}{d\tau}\Big) = 
\cfrac{d}{d\tau}\Big(\cfrac{d\phi^n}{dt}\Big)\cfrac{dt}{d\tau} + \cfrac{d\phi^n}{dt}\cfrac{d^2t}{d\tau^2} = 
\cfrac{d}{dt}\Big(\cfrac{d\phi^n}{dt}\Big)\Big(\cfrac{dt}{d\tau}\Big)^2 = \cfrac{d^2\phi^n}{dt^2}\Big(\cfrac{dt}{d\tau}\Big)^2
$$
for $n=1,2,3$, neglecting second-order infinitesimals.
Therefore applying the above to the (\ref{acce}) we get
$$
\cfrac{d^2\phi^n}{dt^2} = -\Gamma^n_{00} \text{ for } n=1,2,3.
$$

so 

$$
\cfrac{d^2\phi^n}{dt^2} = -\Gamma^n_{00} = -\frac{1}{2} g^{n\rho} \left( \frac{\partial g_{\rho 0}}{\partial x^0} + \frac{\partial g_{\rho 0}}{\partial x^0} - \frac{\partial g_{0 0}}{\partial x^\rho} \right).
$$

and by Assumption \ref{con_time}, neglecting second-order infinitesimals:

$$
\cfrac{d^2\phi^n}{dt^2} = \frac{1}{2} \frac{\partial g_{0 0}}{\partial x^n}.
$$

Since $R_{\mu\nu} = 0$, neglecting second-order infinitesimals we have:
$$
\cfrac{\partial \Gamma^\alpha_{\mu\alpha}}{\partial x^\nu}
- \cfrac{\partial \Gamma^\alpha_{\mu\nu}}{\partial x^\alpha} = 0
$$

Note that with neglecting of second-order infinitesimals,
$$
\cfrac{\partial \Gamma^\alpha_{\mu\alpha}}{\partial x^\nu} = \cfrac{1}{2}g^{\alpha\rho}\Big(
\cfrac{\partial^2 g_{\rho \mu}}{\partial x^\alpha \partial x^\nu} 
+ \cfrac{\partial^2 g_{\rho \alpha}}{\partial x^\mu \partial x^\nu} 
- \cfrac{\partial^2 g_{\mu\alpha}}{\partial x^\alpha \partial x^\nu}\Big),
$$

$$
\cfrac{\partial \Gamma^\alpha_{\mu\nu}}{\partial x^\nu} = \cfrac{1}{2}g^{\alpha\rho}\Big(
\cfrac{\partial^2 g_{\rho \mu}}{\partial x^\nu \partial x^\alpha} 
+ \cfrac{\partial^2 g_{\rho \nu}}{\partial x^\mu \partial x^\alpha} 
- \cfrac{\partial^2 g_{\mu\nu}}{\partial x^\rho \partial x^\alpha}\Big).
$$

Hence,
$$
\cfrac{1}{2}g^{\alpha\rho}
\Big(
\cfrac{\partial^2 g_{\rho \nu}}{\partial x^\mu \partial x^\alpha} 
- \cfrac{\partial^2 g_{\mu\alpha}}{\partial x^\alpha \partial x^\nu}
- \cfrac{\partial^2 g_{\rho \nu}}{\partial x^\mu \partial x^\alpha}
+ \cfrac{\partial^2 g_{\mu\nu}}{\partial x^\rho \partial x^\alpha}
\Big) = 0.
$$


Choose $\mu=\nu = 0$. Then by Assumption \ref{con_time}
$$
g^{mn} \cfrac{\partial^2 g_{00}}{\partial x^m \partial x^n} = 0 \text{ for } n,m=1,2,3.
$$

By Proposition \ref{weak_curv} with neglecting second-order infinitesimals we have
$$
\eta_{nn} \cfrac{\partial^2 g_{00}}{\partial x^n \partial x^n} = 0  \text{ for } n=1,2,3.
$$

Which is a Laplace equation in $\R^3$.
\subsection{Four-velocity}
Assume that we describe a particle in a frame of reference $x^\mu$ as $(\phi^0(t) = t, \phi^1(t), \phi^2(t), \phi^3(t)).$ We define a four-velocity vector as:
$$
\boxed{
v^\mu = \cfrac{d\phi^\mu}{d\tau}
}
$$
where $d\tau$ is a infinitesimal length element of curve $\phi$.\\

Note that 
\begin{equation}\label{infi_shift}
d\tau = (g_{\mu\nu}d\phi^\mu d\phi^\nu)^\frac{1}{2}.
\end{equation}
Hence
$$
\cfrac{d\tau}{dt} = (g_{\mu\nu}\frac{d\phi^\mu}{dt}\frac{d\phi^\nu}{dt})^\frac{1}{2}.
$$
Notive that
$$
v^0\cfrac{d\tau}{dt} = \cfrac{d\phi^0}{d\tau}\cfrac{d\tau}{dt} = \cfrac{d\phi^0}{dt} = 1.
$$
So
$$
\boxed
{
v^0 = (g_{\mu\nu}\frac{d\phi^\mu}{dt}\frac{d\phi^\nu}{dt})^{-\frac{1}{2}}
}
$$
Note that $v^0$ is what we usually denote in literature as $\gamma$.
On the other hand:
$$
\cfrac{d\tau}{dt} = (v^0)^{-1}
$$
And thus:
$$
\cfrac{d\phi^\mu}{dt} = \cfrac{d\phi^\mu}{d\tau}\cfrac{d\tau}{dt} = v^1(v^0)^{-1}. 
$$
$$
\boxed{
\cfrac{d\phi^\mu}{dt} = v^1(v^0)^{-1}
}
$$
Which means that once you know the four-velocity in given frame of reference, you know space velocities as well.

There is one more implication from (\ref{infi_shift}):
\begin{equation}\label{one}
\boxed{
g_{\mu\nu}v^\mu v^\nu = 1
}
\end{equation}

We will show that $v^\mu$ is a tensor.
Assume that $y^\mu$ is a new frame of reference. Note that
$$
d\hat{\phi}^\mu = \cfrac{\partial y^\mu}{\partial x^\nu} d\phi^\nu.
$$
Since $d\tau$ is invariant $\hat{v}^\mu d\tau = \cfrac{\partial y^\mu}{\partial x^\nu} v^\nu d\tau$, hence
$$
\boxed{
\hat{v}^\mu = \cfrac{\partial y^\mu}{\partial x^\nu} v^\nu
}
$$
Let $\omega^\mu = \cfrac{d\phi^\mu}{dt}$. ($\omega^\nu$ is not a tensor). Then $\omega^\mu = v^\mu(v^0)^{-1}$.\\
Note that usually encountered in literature $\beta = (\omega^1 + \omega^2 + \omega^3)^{\frac{1}{2}}$. Note
\begin{equation}
\boxed{
v^0 = (1 - \beta^2)^{-\frac{1}{2}}}
\end{equation} 
To simplify calculations, we can assume that $g_{\mu\nu} = \eta_{\mu\nu}$ at $(0,0,0,0)$. $\phi^\mu(0) = 0$ and that particle at the moment $t=0$ is moving along $x^1$, i.e. $\omega^2 = \omega^3 = 0$. Thus $v^2 = v^3 = 0$.
Let 
\begin{equation}\label{lo1}
\begin{cases}
y^0 = v^0x^0 - v^1x^1\\
y^1 = v^0x^1 - v^1x^0\\
y^2 = x^2\\
y^3 = x^3
\end{cases}
\end{equation}
It's easy to show that $\hat{v}^0 = 1$ and $\hat{v}^1 = \hat{v}^2 = \hat{v}^3 = 0$, which simply means that for $t = 0$ the particle is in rest in the frame of reference $y^\mu$.
Obviously
\begin{equation}\label{lo2}
\begin{cases}
x^0 = v^0y^0 + v^1y^1\\
x^1 = v^0y^1 + v^1y^0\\
x^2 = y^2\\
x^3 = y^3
\end{cases}
\end{equation}

$$
\hat{g}_{00} = g_{\mu\nu} \cfrac{\partial x^\mu}{\partial y^0}\cfrac{\partial x^\nu}{\partial y^0} = 
(v^0)^2 - (v^1)^2 = 1,
$$
$$
\hat{g}_{01} = g_{\mu\nu} \cfrac{\partial x^\mu}{\partial y^0}\cfrac{\partial x^\nu}{\partial y^1} = 
v^0v^1 - v^1v^0 = 0,
$$
$$
\hat{g}_{11} = g_{\mu\nu} \cfrac{\partial x^\mu}{\partial y^1}\cfrac{\partial x^\nu}{\partial y^1} = 
(v^1)^2 - (v^0)^2 = -1.
$$

Thus also $\hat{g}_{\mu\nu} = \eta_{\mu\nu}$ for in $(0,0,0,0)$.

Consider now particle $\phi$ in some very short (in fact as short as we need) time after $t=0$. We can assume that in the frame of reference $x$ we can describe it as:
$$t \to (t, \omega^1t, 0, 0)$$.

Let's see this particle in the frame $y$. Assume that time interval that we consider is so short that we all the time can use (\ref{lo1}):

$$
t \to (v^0t - v^1w^1t, v^0w^1t - v^1t, 0, 0) = (v^0(1-(w^1)^2)t, 0, 0, 0) =
((v^0)^{-1}t, 0, 0, 0)
$$
It follows that if for the observer in $x$ interval $\Delta t$ passed, for observer in $y$ interval $(v^0)^{-1}\Delta t$ passed, which is of course time dilatation.  
Note that $t$ in frame of reference $y$ is merely a parameter.
To summarise, this is the law of time dilatation
\begin{equation}
\boxed{\Delta t' = (v^0)^{-1}\Delta t}
\end{equation}
Where is $\Delta t$ is time passed measured by an observer who is in rest relative to frame of reference $x^\mu$ and $\Delta t'$ is time passed measured by an observer who is in rest relative to a particle frame of reference $y^\mu$. 

Now consider second particle that is moving in exactly the same direction as first one but proceeding it with a distance $\Delta x'$ in frame of reference $x$. We can describe it as:
$$
t \to (t, w^1t + \Delta x', 0, 0).
$$
Now look at this in frame of reference $y$.
$$
t \to ((v^0)^{-1}t - v^1\Delta x', v^0\Delta x', 0, 0).
$$
So the distance from particles in frame of reference $y$ is $v^0\Delta x'$.
To summarise, this is the law of length contraction
\begin{equation}
\boxed{
\Delta x' = (v^0)^{-1} \Delta x
}
\end{equation}
Where $\Delta x$ is a rest distance between particles and $\Delta x'$ is a distance between particles measured by an observer who is in rest in the frame of refernce $x^\mu$.
\begin{theorem}
\label{metric_inter}
If $u^\mu$, $v^\mu$ are 4-velocities of two observers at the point of intersection of their geodesics, then 
\begin{equation}
\label{energy_coef}
g_{\mu\nu} u^\mu v^\nu = (1 - \omega)^{-\frac{1}{2}}
\end{equation}
where $\omega$ is a value of the relative velocity measured by observers,
\begin{equation}
\label{rel_vel}
g_{\mu\nu} x^\mu v^\nu = -(1 - \omega)^{-\frac{1}{2}}\omega_x.
\end{equation}
where $x_\mu$ is a vector of length $-1$, orthogonal to $u_\mu$, and $\omega_x$ is a velocity of $v_\nu$ measured by an observer $u^\mu$ along the vector $x_\mu$.
\end{theorem}
\begin{proof}
Let $p$ be the point of intersection. As the value of $g_{\mu\nu} u^\mu v^\nu$ doesn't depend on frame of reference, we are free to choose any frame of reference we need. Let's choose the frame of reference where $u = (1, 0, 0, 0)$ and $v=(v^0,v^1,v^2,v^3)$ and $g_{\mu\nu} = \eta_{\mu\nu}$ at point $p$. Such frame of reference will be just Riemannian coordinates with $0-axis$ along vector $u^\mu$. Thus $g_{\mu\nu} u^\mu v^\nu = \eta_{00} v_0 = v_0$, which proves equation \ref{energy_coef}.
We may also require that in the frame of reference where $u = (1, 0, 0, 0)$, $x = (0,1,0,0)$. Thus $g_{\mu\nu} x^\mu v^\nu = - v_1 = -v_0\omega_1$. which proves equation \ref{rel_vel}.
\end{proof}

\begin{fact}
If $u^\mu$, $v^\nu$ are 4-velocities of two observers at the point of intersection of their geodesics and
 $\lambda$ is an arbitrary density of some quantity measured locally by an observer $v^\nu$, then 
$\lambda g_{\mu\nu} u^\mu v^\nu$ is a density of this quantity measured locally by an observer $u^\mu$. 
\end{fact}

\begin{fact}
If $p^\nu$ is a 4-momentum of a particle and $u^\mu$ is a 4-velocity of an observer at the point of interection with the geodesic of the particle, then
\begin{equation}
E = g_{\mu\nu} u^\mu p^\nu, 
\end{equation}

\begin{equation}
p_x = g_{\mu\nu} x^\mu p^\nu.
\end{equation}
where $E$ is an energy of the particle measured by an observer, $x_\mu$ is a vector of length $-1$, orthogonal to $u_\mu$, and $p_x$ is a momentum of a particle measured by an observer $u^\mu$ along the vector $x_\mu$.  
\end{fact}

\subsection{Static spacetime}

\begin{definition}
We will say that system of coordinates $x$ is static if $\cfrac{\partial g_{\mu\nu}}{\partial x^0} = 0$ and $g_{m0} = 0$ for $m=1,2,3$.
\end{definition}

\begin{fact}
If the system of coordinates $x$ is static, then
\begin{equation}
g^{00} = (g_{00})^{-1},
\end{equation}
\begin{equation}
g^{n0} = 0,
\end{equation}
\begin{equation}
g_{na}g^{am} = \delta^m_n,
\end{equation} 
\begin{equation}
\Gamma^a_{00} = - \cfrac{1}{2} g^{ab}\cfrac{\partial g_{00}}{\partial x^b},
\end{equation}
\begin{equation}
\Gamma^a_{nm} = \frac{1}{2} g^{ab} \left( \frac{\partial g_{bm}}{\partial x^n} + \frac{\partial g_{bn}}{\partial x^m} - \frac{\partial g_{mn}}{\partial x^b} \right),
\end{equation}

\begin{equation}
\Gamma^a_{n0} = \Gamma^0_{00} = 0.
\end{equation}
\begin{proof}
$$\Gamma^a_{00} = {1 \over 2} g^{ab} \left( {\partial g_{b0} \over \partial x^0} + {\partial g_{b0} \over \partial x^0} - {\partial g_{00} \over \partial x^b} \right) = - \cfrac{1}{2} g^{ab}\cfrac{\partial g_{00}}{\partial x^b}.$$

\begin{gather*}
\Gamma^a_{nm} = {1 \over 2} g^{ab} \left( {\partial g_{bm} \over \partial x^n} + {\partial g_{bn} \over \partial x^m} - {\partial g_{mn} \over \partial x^b}\right) + {1 \over 2} g^{a0} \left( \cdots \right)  = \\
{1 \over 2} g^{ab} \left( {\partial g_{bm} \over \partial x^n} + {\partial g_{bn} \over \partial x^m} - {\partial g_{mn} \over \partial x^b}\right).
\end{gather*}

$$
\Gamma^a_{n0} = {1 \over 2} g^{ab} \left( {\partial g_{b0} \over \partial x^n} + {\partial g_{bn} \over \partial x^0} - {\partial g_{0n} \over \partial x^b} \right) + {1 \over 2}g^{a0}\left(\cdots \right) = 0.
$$
\end{proof}

Note that condition $g_{n0} = 0$ provides that $(dt, 0, 0, 0)$ and $(0,dx^1, dx^2, dx^3)$ are orthogonal at each point. We may say that all static observers agree locally on what is their space. Motivated by this we can define space as $t = const$.

\end{fact}

\begin{theorem}
\label{space_est}
If $x$ is a static coordinate system and $t\to(t,\phi_1(t), \phi_2(t), \phi_3(t))$ is an equation of free particle (geodesic in coordinates $x$) and $\omega_\nu = {d\phi^\nu \over dt}$, then
\begin{equation}
\omega^a \nabla_a \omega^b = - \Gamma^a_{00},
\end{equation}
\begin{equation}
\text{i.e.  } {d^2\phi^a \over dt^2} + \Gamma^a_{nm} {d\phi^n \over dt} {d\phi^m \over dt} = - \Gamma^a_{00}.
\end{equation}
where $\nabla$ is a covariant derivative in 3 dimensional manifold with metric $g_{nm}$.
\end{theorem}

\begin{theorem}
If $x$ is a static coordinate system and $t\to(t,\phi_1(t), \phi_2(t), \phi_3(t))$ is a null geodesic in coordinates $x$, then for $t\to(\phi_1(t), \phi_2(t), \phi_3(t))$, we have $\delta\int dt = 0$.
\end{theorem}
\begin{proof}
Let's define a new metric $H_{\mu\nu} = g_{\mu\nu}(g_{00})^{-1}$. $H$ is conformally related to $g$. Thus they have the same null geodesics. Note that $x$ is a static coordinate system in $H$. Then by Theorem \ref{space_est}

\begin{equation}
{d^2\phi^a \over dt^2} + \overset{H}{\Gamma^a_{nm}} {d\phi^n \over dt} {d\phi^m \over dt} = - \overset{H}{\Gamma^a_{00}}.
\end{equation} 
But $\overset{H}{\Gamma^a_{00}} = 0$. Then $t\to(\phi_1(t), \phi_2(t), \phi_3(t))$ is a geodesic equation in a 3 dimensional manifold with metric $H_{mn}$. So $\delta \int ds_H = 0$. Note that
$$
ds^2_H = - H_{mn}dx^m dx^n = - (g_{00})^{-1}g_{mn}dx^m dx^n = (g_{00})^{-1} ds^2.
$$ 
Hence
\begin{equation}
\label{var_met}
\delta \int g_{00}^{- {1 \over 2}}ds = 0.
\end{equation}
Because we are on the null geodesic in $x$ with metric $g$, we have

$$
0 = g_{00}dt^2 + g_{nm}dx^ndx^m = g_{00} dt^2 - ds^2.
$$

So
$$
dt = (g_{00})^{- {1 \over 2}}ds.
$$
\end{proof}
And \ref{var_met} becomes
$$
\delta\int dt = 0.
$$
\subsection{Stress-energy tensor}

We will construct some illustrative example of the stress-energy tensor. Assume that we have distribution of matter whose velocity varies continuously from one point to a neighboring one. Let $\rho$ be a scalar field of rest density measured in the rest frame of reference of an infinitesimal part of the matter distribution. Let $v^\mu$ be a vector field of 4-velocities of an infinitesimal part of the matter distribution at the point. Let define
\begin{equation}
T^{\mu\nu} = \rho v^\mu v^\nu.
\end{equation}
Let $\omega$ be a relative velocity between observer $u^\mu$ and an infenitesimal matter element $v^\mu$ at each point.
Note that rest density of the matter distribution is just some quantity distributed in space. Forget for a moment that this is a rest density, let's treat this as some kind of abstract quantity. Then
\begin{equation}
\underbrace{\rho (1 - \omega)^{-\frac{1}{2}}}_{\text{rest density} * \frac{1}{\text{length contraction}}}
\end{equation}
is a density of this quantity measured by the observer $u^\mu$, i. e. density of rest density measured by the observer $u^\mu$.
\begin{fact}
$T_{\mu\nu} u^\mu u^\nu$ is a energy density measured locally by an observer $u^\nu$.
\end{fact}
\begin{proof}
We will use notation from Theorem \ref{metric_inter}
\begin{equation}
T_{\mu\nu} u^\mu u^\nu = \rho v^\alpha v^\beta g_{\mu\alpha}g_{\mu\beta} u^\mu u^\nu = \overbrace{\underbrace{\rho (1 - \omega)^{-\frac{1}{2}}}_{\text{rest density} * \frac{1}{\text{length contraction}}} \cdot(1 - \omega)^{-\frac{1}{2}}}^\text{relative energy density}.
\end{equation}
\end{proof}

\begin{fact}
$T_{\mu\nu} x^\mu u^\nu$ is a density of momentum along the vector $x^\mu$, measured locally by an observer $u^\nu$ -- where $x^\mu$ is an orthogonal to $u^\mu$ vector of length $-1$.  
\end{fact}
\begin{proof}

\begin{equation}
T_{\mu\nu} x^\mu u^\nu = \rho v^\alpha v^\beta g_{\mu\alpha}g_{\mu\beta} x^\mu u^\nu = \overbrace{\underbrace{\rho (1 - \omega)^{-\frac{1}{2}}}_{\text{rest density} * \frac{1}{\text{length contraction}}} \cdot (1 - \omega)^{-\frac{1}{2}}\omega_x}^\text{relative momentum density}.
\end{equation}
\end{proof}

\begin{fact}
$T_{\mu\nu} u^\mu x^\nu$ is a energy flux across the surface with normal vector $x^\mu$, measured locally by an observer $u^\nu$ -- where $x^\mu$ is an orthogonal to $u^\mu$ vector of length $-1$.  
\end{fact}
\begin{proof}
\begin{equation}
T_{\mu\nu} u^\mu x^\nu = \rho v^\alpha v^\beta g_{\mu\alpha}g_{\mu\beta} u^\mu x^\nu = \overbrace{\underbrace{\rho (1 - \omega)^{-\frac{1}{2}}}_{\text{rest density} * \frac{1}{\text{length contraction}}} \cdot (1 - \omega)^{-\frac{1}{2}}}^\text{relative energy density} \omega_x.
\end{equation}
\end{proof}

Of course $T_{\mu\nu} x^\mu u^\nu = T_{\mu\nu} u^\mu x^\nu$.

\begin{fact}
$T_{\mu\nu} x^\mu y^\nu$ is a flux of momentum along $x^\mu$ across the surface with normal vector $y^\mu$, measured locally by an observer $u^\nu$ -- where $x^\mu, y^\nu$ are orthogonal to $u^\mu$ vectors of length $-1$.  
\end{fact}
\begin{proof}
\begin{equation}
T_{\mu\nu} x^\mu y^\nu = \rho v^\alpha v^\beta g_{\mu\alpha}g_{\mu\beta} u^\mu x^\nu = \overbrace{\underbrace{\rho (1 - \omega)^{-\frac{1}{2}}}_{\text{rest density} * \frac{1}{\text{length contraction}}} \cdot (1 - \omega)^{-\frac{1}{2}}\omega_x}^\text{relative momentum density} \omega_y.
\end{equation}
\end{proof}   
\end{document}