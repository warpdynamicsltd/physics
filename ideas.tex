\documentclass[main.tex]{subfiles}
\begin{document}
\section{Quantum Physics}
\subsection{Mathematical formalism}
There are many problems in Quantum Mechanics which relates to imprecise use of mathematical apparatus, which cause at least propaedeutical problems. Some of them are described in an excellent paper \cite{gieres2000}. On the other hand it is well known fact that quantum mechanics done on finite dimensional Hilbert spaces is more adoptable mathematically to comprehend. Isn't any way to make Quantum Mechanics in some kind of limit space of finite dimentinal Hilber spaces? 
\subsection{Conservation of wave function modulus}
In finite dimension Hilbert space it is relatively easy to prove, using a finite spectral theorem, that if observable $A$ and $H$ commute than for any quantum state which evolves according to Schrödinger equation
\begin{equation}
    i\frac{d\psi}{dt} = H\psi.
\end{equation}
\begin{equation}
\langle A\psi(t), e_i \rangle = \text{const},
\end{equation}
for any eigenvector $e_i$ of $A$. How this translates to the infinite dimensional case to have
\begin{equation}
    |\hat{\psi}(t)|^2 = \text{const}
\end{equation}
when momentum operator $P$ commutes with $H$. Does it take for $P$ and $H$ to have the same spectral basis? In case of free particle it holds (see equation \ref{free-momentum-representation})
\end{document}